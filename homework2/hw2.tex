\documentclass[times]{article}

\usepackage{graphicx}
\usepackage{placeins}
\usepackage[none]{hyphenat}
\usepackage{amsmath}
\usepackage[us]{datetime}
\usepackage[explicit]{titlesec}
\titleformat{\section}{\normalfont}{}{0em}{\textbf{\large Problem  \thesection}\  \normalsize #1}
\begin{document}
	\title{CS 5800 Distributed OS - Spring 2017 - Homework 2}
	\author{Josh Herman \\ Dalton Cole \\ Neil Blood}
	\date{\formatdate{22}{2}{2017}}
	\maketitle

	\section{Problem 1}

	\section{Problem 2}

	The program is guaranteed to finish in n steps. This is because, in the worst case, x[n-1] = t. This is true because t is guaranteed to be in the set. In other words: \\
	Set s = \{\} \\
	the max size of s will be n. For every x[i] $\not=$ t, append x[i] to s. This will represent all of the elements before t in the list. The max size of the set is n, which is the maximum number of steps to terminate the program.

	\section{Problem 3}

	Safety: When two or more users edit the same document at the same time, their edits are never lost.

	Liveliness: Document will auto-save eventually, saving the data.

	\section{Problem 4}

	\begin{center}
		\begin{tabular}{ |c|c|c|c|c|c| } 
			 \hline
			 State 	& Action 		& c[0] 	& c[1] 	& c[2] 	& c[3] \\ 
			 \hline
			 A 		& 				& 0		& 0 	& 2 	& 0 	\\ 
			 B 		& $P_0$ moves	& 2		& 0 	& 2 	& 0 	\\ 
			 C 		& $P_0$ moves	& 0		& 0 	& 2 	& 0 	\\ 
			 \hline
		\end{tabular}
	\end{center}

	As can be seen from the above table, states A and C are identical, thus program can cycle with this given input, thus not guaranteeing termination.

	\section{Problem 5}

	\section{Problem 6}

	\begin{center}
		\begin{tabular}{ |c c c c c| } 
			\hline
			2 	& 0	& 1	& 0	& 2	\\ 
			\hline
			1 	& 2	& 2	& 2	& 1	\\ 
			\hline
			0 	& 0	& 0	& 0	& 0	\\ 
			\hline
		\end{tabular}
	\end{center}
	
	
		
\end{document}